\documentclass[journal,12pt,twocolumn]{IEEEtran}

\usepackage{setspace}
\usepackage{gensymb}

\singlespacing


\usepackage[cmex10]{amsmath}

\usepackage{amsthm}

\usepackage{mathrsfs}
\usepackage{txfonts}
\usepackage{stfloats}
\usepackage{bm}
\usepackage{cite}
\usepackage{cases}
\usepackage{subfig}

\usepackage{longtable}
\usepackage{multirow}

\usepackage{enumitem}
\usepackage{mathtools}
\usepackage{steinmetz}
\usepackage{tikz}
\usepackage{circuitikz}
\usepackage{verbatim}
\usepackage{tfrupee}
\usepackage[breaklinks=true]{hyperref}
\usepackage{graphicx}
\usepackage{tkz-euclide}

\usetikzlibrary{calc,math}
\usepackage{listings}
    \usepackage{color}                                            %%
    \usepackage{array}                                            %%
    \usepackage{longtable}                                        %%
    \usepackage{calc}                                             %%
    \usepackage{multirow}                                         %%
    \usepackage{hhline}                                           %%
    \usepackage{ifthen}                                           %%
    \usepackage{lscape}     
\usepackage{multicol}
\usepackage{chngcntr}

\DeclareMathOperator*{\Res}{Res}

\renewcommand\thesection{\arabic{section}}
\renewcommand\thesubsection{\thesection.\arabic{subsection}}
\renewcommand\thesubsubsection{\thesubsection.\arabic{subsubsection}}

\renewcommand\thesectiondis{\arabic{section}}
\renewcommand\thesubsectiondis{\thesectiondis.\arabic{subsection}}
\renewcommand\thesubsubsectiondis{\thesubsectiondis.\arabic{subsubsection}}


\hyphenation{op-tical net-works semi-conduc-tor}
\def\inputGnumericTable{}                                 %%

\lstset{
%language=C,
frame=single, 
breaklines=true,
columns=fullflexible
}
\begin{document}


\newtheorem{theorem}{Theorem}[section]
\newtheorem{problem}{Problem}
\newtheorem{proposition}{Proposition}[section]
\newtheorem{lemma}{Lemma}[section]
\newtheorem{corollary}[theorem]{Corollary}
\newtheorem{example}{Example}[section]
\newtheorem{definition}[problem]{Definition}

\newcommand{\BEQA}{\begin{eqnarray}}
\newcommand{\EEQA}{\end{eqnarray}}
\newcommand{\define}{\stackrel{\triangle}{=}}
\bibliographystyle{IEEEtran}

\providecommand{\mbf}{\mathbf}
\providecommand{\pr}[1]{\ensuremath{\Pr\left(#1\right)}}
\providecommand{\qfunc}[1]{\ensuremath{Q\left(#1\right)}}
\providecommand{\sbrak}[1]{\ensuremath{{}\left[#1\right]}}
\providecommand{\lsbrak}[1]{\ensuremath{{}\left[#1\right.}}
\providecommand{\rsbrak}[1]{\ensuremath{{}\left.#1\right]}}
\providecommand{\brak}[1]{\ensuremath{\left(#1\right)}}
\providecommand{\lbrak}[1]{\ensuremath{\left(#1\right.}}
\providecommand{\rbrak}[1]{\ensuremath{\left.#1\right)}}
\providecommand{\cbrak}[1]{\ensuremath{\left\{#1\right\}}}
\providecommand{\lcbrak}[1]{\ensuremath{\left\{#1\right.}}
\providecommand{\rcbrak}[1]{\ensuremath{\left.#1\right\}}}
\theoremstyle{remark}
\newtheorem{rem}{Remark}
\newcommand{\sgn}{\mathop{\mathrm{sgn}}}
\providecommand{\abs}[1]{\left\vert#1\right\vert}
\providecommand{\res}[1]{\Res\displaylimits_{#1}} 
\providecommand{\norm}[1]{\left\lVert#1\right\rVert}
%\providecommand{\norm}[1]{\lVert#1\rVert}
\providecommand{\mtx}[1]{\mathbf{#1}}
\providecommand{\mean}[1]{E\left[ #1 \right]}
\providecommand{\fourier}{\overset{\mathcal{F}}{ \rightleftharpoons}}
%\providecommand{\hilbert}{\overset{\mathcal{H}}{ \rightleftharpoons}}
\providecommand{\system}{\overset{\mathcal{H}}{ \longleftrightarrow}}
	%\newcommand{\solution}[2]{\textbf{Solution:}{#1}}
\newcommand{\solution}{\noindent \textbf{Solution: }}
\newcommand{\cosec}{\,\text{cosec}\,}
\providecommand{\dec}[2]{\ensuremath{\overset{#1}{\underset{#2}{\gtrless}}}}
\newcommand{\myvec}[1]{\ensuremath{\begin{pmatrix}#1\end{pmatrix}}}
\newcommand{\mydet}[1]{\ensuremath{\begin{vmatrix}#1\end{vmatrix}}}

\numberwithin{equation}{subsection}

\makeatletter
\@addtoreset{figure}{problem}
\makeatother
\let\StandardTheFigure\thefigure
\let\vec\mathbf

\renewcommand{\thefigure}{\theproblem}

\def\putbox#1#2#3{\makebox[0in][l]{\makebox[#1][l]{}\raisebox{\baselineskip}[0in][0in]{\raisebox{#2}[0in][0in]{#3}}}}
     \def\rightbox#1{\makebox[0in][r]{#1}}
     \def\centbox#1{\makebox[0in]{#1}}
     \def\topbox#1{\raisebox{-\baselineskip}[0in][0in]{#1}}
     \def\midbox#1{\raisebox{-0.5\baselineskip}[0in][0in]{#1}}
\vspace{3cm}
\title{Assignment 19}
\author{KUSUMA PRIYA\\EE20MTECH11007}

\maketitle
\newpage

\bigskip
\renewcommand{\thefigure}{\theenumi}
\renewcommand{\thetable}{\theenumi}
Download codes from 
%
\begin{lstlisting}
https://github.com/KUSUMAPRIYAPULAVARTY/assignment19
\end{lstlisting}
%
 
\section{QUESTION}
Let $\vec{V}$ be the vector space of all polynomial functions over the field of real numbers.Let $a$ and $b$ be fixed real numbers and let $f$ be the linear functional on $\vec{V}$ defined by 
\begin{align}
    f(p)=\int_a ^b p(x) \,dx
\end{align}
If $D$ is the differentiator operator on $\vec{V}$, what is $D^tf$?
%

\section{Solution}
\begin{table}[!h]
\centering
\begin{tabular}{|p{3cm}|p{5cm}|}
\hline
\textbf{PARAMETERS}&\textbf{DESCRIPTION}\\
\hline
$\mathbb{R}$&Field of real numbers\\
\hline
$\vec{V}$&Vector space of all polynomials over $\mathbb{R}$\\
\hline
$a,b$ & Fixed real numbers\\
\hline
$f$ &Linear functional on $\vec{V}$\\
\hline
$D$ & Differentiator operator on $\vec{V}$\\
\hline
$D^t$ &Transpose of $D$\\
\hline
\end{tabular}
\caption{Input Parameters}
\end{table}



\begin{table*}[!t]
    \centering
    \begin{tabular}{|l|l|}
    \hline
    \textbf{STATEMENTS} & \textbf{DERIVATIONS}\\
    \hline
      $D^t$ is transpose of $D$ & 
      \parbox{10cm}{\begin{align}
         (D^tf)(p)=f[D(p)]
      \end{align}}\\
      \hline
      A polynomial of degree $n+1$ & \\
       $ p(x)=c_0+c_1x+c_2x^2+\hdots+c_nx^n$ &
       \parbox{10cm}{\begin{align}
         p=\vec{c}^T\vec{x}\\
         \vec{c}=\myvec{c_0\\c_1\\c_2\\\vdots\\c_n},
         \vec{x}=\myvec{1\\x\\x^2\\\vdots\\x^n}
       \end{align}}\\
       \hline
       $ f(p)=\int_a ^b p(x) \,dx$& 
        \parbox{10cm}{\begin{align}
          f(p)=\vec{c}^T\vec{F}\\
          \vec{F}=\myvec{b-a\\\frac{b^2-a^2}{2}\\\frac{b^3-a^3}{3}\\\vdots\\\frac{b^{n+1}-a^{n+1}}{n+1}}
       \end{align}}\\
       \hline
      $D(p)=c_1+2c_2x+3c_3x^2+\hdots+nc_nx^{n-1}$ &
        \parbox{10cm}{\begin{align}
          D(p)=\vec{c}^T\vec{D}\vec{x}\\
          \vec{D}=\myvec{0&0&0&\hdots&0&0\\1&0&0&\hdots&0&0\\0&2&0&\hdots&0&0\\0&0&3&\hdots&0&0\\\vdots&\vdots&\vdots&\hdots&\vdots&\vdots\\0&0&0&\hdots&n&0}
       \end{align}}\\
       \hline
            $f[D(p)]=c_1(b-a)+c_2(b^2-a^2)+\hdots+c_n(b^n-a^n)$
      & 
        \parbox{10cm}{\begin{align}
        \text{Let  }D(p)=p'\\
         \implies p'=\vec{c'}^T\vec{x}\\
         \text{where }\vec{c'}^T=\myvec{c_1\\2c_2\\3c_3\\\vdots\\nc_n\\0}\\
         f[D(p)]=\vec{c'}^T\vec{F}
       \end{align}}\\
       \hline
      $(D^t f)(p)=p(b)-p(a)$ & 
         \parbox{10cm}{\begin{align}
         (D^t f)(p)=c_1(b-a)+c_2(b^2-a^2)+\\\hdots+c_n(b^n-a^n)+c_0-c_0\\
         =(c_0+c_1b+c_2b^2+\hdots+c_nb^n)-\\
         (c_0+c_1a+c_2a^2+\hdots+c_na^n)\\
         =p(b)-p(a)
       \end{align}}\\
       \hline
    \end{tabular}
    \caption*{TABLE1: Proof}
\end{table*}
\end{document}


